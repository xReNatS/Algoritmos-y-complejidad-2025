En el mundo informático y en el cotidiano, el papel de los algoritmos es primordial,
pues permiten resolver problemas complejos de manera eficiente y automática. Un
algoritmo puede entenderse como un conjunto de instrucciones que permiten,
mediante datos de entrada, conseguir datos de salida, es decir, es la función que
permite llegar de un conjunto de datos a una solución.
El desarrollo y análisis de algoritmos no solo se enfoca en encontrar soluciones
correctas, sino también en mejorar su rendimiento. Factores como la velocidad de
ejecución, el uso de recursos y la escalabilidad son determinantes para la eficiencia de
un algoritmo. En un mundo cada vez más conectado, donde el volumen de datos y la
complejidad de las tareas crecen de manera exponencial, diseñar algoritmos
eficientes se ha convertido en una prioridad clave.
Este informe tiene como objetivo explorar dos tipos de algoritmos, el de Ordenamiento,
es decir, ordenar los datos numéricos dentro de un vector, y de Multiplicación de
matrices. Para cada tipo se hará uso de varios algoritmos con el mismo fin y se
compararan en distintos escenarios para evaluar su rendimiento y su complejidad.\cite{elsevier_abstract_2024}.

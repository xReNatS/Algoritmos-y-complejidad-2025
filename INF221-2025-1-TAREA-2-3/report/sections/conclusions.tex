Los resultados obtenidos a lo largo del experimento permiten afirmar que el uso de programación dinámica representa una mejora significativa frente a la estrategia de fuerza bruta para resolver el problema de detección de diferencias entre dos secuencias. Esta mejora no solo se refleja en la reducción del tiempo de ejecución, sino también en la capacidad de escalar hacia entradas más grandes sin comprometer la viabilidad computacional del algoritmo.

El análisis comparativo entre ambos enfoques evidencia cómo la elección de una estrategia algorítmica adecuada puede determinar la eficiencia y aplicabilidad de una solución en contextos reales. La fuerza bruta, aunque conceptualmente clara, demuestra ser inviable para volúmenes altos de datos, mientras que la programación dinámica logra mantener un desempeño estable incluso en escenarios más exigentes.

De este modo, los resultados respaldan la necesidad de recurrir a enfoques más estructurados y optimizados para problemas combinatorios, especialmente cuando se espera escalar el uso del algoritmo. La información empírica recolectada no solo valida las complejidades teóricas discutidas, sino que también ilustra con claridad el impacto que tienen en la práctica.

Este trabajo, en definitiva, reafirma la importancia del análisis algorítmico desde una doble perspectiva: teórica y experimental, permitiendo una comprensión más integral del comportamiento de las soluciones frente a distintas condiciones de entrada.

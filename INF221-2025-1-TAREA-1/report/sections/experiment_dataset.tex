Los casos de prueba que fueron utilizados cubren las cuatro dimensiones especificadas para cada problema: tamaño de entrada (\(n\)), tipo de estructura o disposición (\(t\)), dominio de valores (\(d\)) y réplica de muestra (\(m\)). A continuación se comentan ambos grupos de casos.

\subsubsection{Multiplicación de matrices}
\begin{itemize}
  \item \textbf{Tamaños (\(n\))}: \(\{2^4,\,2^6,\,2^8,\,2^{10}\}\).  
    Estos valores permiten medir cómo escalan el tiempo (\(O(n^3)\) vs. \(O(n^{\log_2 7})\)) y el uso de memoria con el crecimiento de la dimensión.
  \item \textbf{Tipos (\(t\))}: \{\emph{densa}, \emph{diagonal}, \emph{dispersa}\}.  
    Cada estructura permite evaluar el rendimiento en escenarios de cómputo completo, optimización para ceros en diagonal y aprovechamiento de la dispersión de datos.
    \begin{itemize}
      \item \emph{Densa}: coeficientes aleatorios no nulos, representa el peor caso de cómputo completo.  
      \item \emph{Diagonal}: matrices con solo la diagonal principal distinta de cero, evalúa cómo afectan las optimizaciones en cero.  
      \item \emph{Dispersa}: pocos elementos no nulos distribuidos aleatoriamente, comprueba si el overhead de recursión amortiza el cálculo en estructuras crestas.  
    \end{itemize}
  \item \textbf{Dominio (\(d\))}: \(\{D0,\,D10\}\), con  
    \(D0=\{0,1\}\) y \(D10=\{0,\dots,9\}\).  
    Esta variación confirma que los algoritmos mantienen su comportamiento sin depender de la amplitud del dominio.
  \item \textbf{Muestras (\(m\))}: \(\{a,b,c\}\).  
    Tres réplicas para cada combinación permiten calcular varianza y construir intervalos de confianza de tiempo y memoria.
  \item \textbf{Generación}: los archivos fueron producidos con \texttt{matrix\_generator.py} 
  
  (en \texttt{code/matrix\_multiplication/scripts/}), de manera que se pueda reproducir lo experimentado.
\end{itemize}

\subsubsection{Ordenamiento}
\begin{itemize}
  \item \textbf{Tamaños (\(n\))}: \(\{10^1,\,10^3,\,10^5,\,10^7\}\).  
    Con ellos se mide el desempeño desde vectores muy pequeños hasta grandes, comparando \(O(n\log n)\) frente a \(O(n^2)\).
  \item \textbf{Tipos (\(t\))}: \{\emph{aleatorio}, \emph{ascendente}, \emph{descendente}\}.  
    Cada disposición corresponde a caso promedio, mejor y peor, permitiendo analizar la robustez de los algoritmos.
  \item \textbf{Dominio (\(d\))}: \(\{D1,\,D7\}\), donde  
    \(D1=\{0,1\}\) y \(D7=\{0,\dots,10^7\}\).  
    Esto asegura que las comparaciones no se vean afectadas por la dispersión de valores.
  \item \textbf{Muestras (\(m\))}: \(\{a,b,c\}\).  
    Tres instancias aleatorias para cada caso permiten estimar la variabilidad estadística de tiempos y consumos.
  \item \textbf{Generación}: los vectores fueron creados con \texttt{array\_generator.py} (en \texttt{code/sorting/scripts/}), asegurando que las pruebas coincidan con el repositorio oficial.
\end{itemize}

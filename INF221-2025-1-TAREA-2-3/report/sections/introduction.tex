El an\'alisis y dise\~no de algoritmos es una de las bases fundamentales en la formaci\'on de estudiantes en Ciencias de la Computaci\'on. Comprender c\'omo se comportan distintos algoritmos en la pr\'actica permite tomar mejores decisiones sobre cu\'ando y por qu\'e utilizar uno u otro. En este trabajo se busca evaluar el rendimiento de dos enfoques para resolver un problema cl\'asico: encontrar las diferencias entre dos secuencias.

El problema de encontrar las diferencias entre dos secuencias consiste en identificar las partes no coincidentes entre ambas, resaltando al mismo tiempo las similitudes. Para lograr esto, se suele buscar la subsecuencia común más larga (LCS), y luego, con base en esa referencia, segmentar las porciones que no coinciden. Esta tarea es útil en áreas como el control de versiones, la comparación de textos o secuencias biológicas, donde es importante entender cómo y dónde dos cadenas difieren.

Para ello, se comparan dos estrategias distintas: una implementaci\'on por fuerza bruta, que explora todas las posibles subsecuencias para encontrar la subsecuencia com\'un m\'as larga, y otra basada en programaci\'on din\'amica, que utiliza una soluci\'on eficiente construyendo una tabla de subproblemas. Ambos enfoques fueron implementados y puestos a prueba en diferentes casos.

El objetivo principal de este informe es analizar cu\'ales son las diferencias pr\'acticas entre ambos algoritmos, en t\'erminos de tiempo de ejecuci\'on y uso de memoria. Para ello, se desarroll\'o una serie de experimentos que permiten observar hasta qu\'e punto es viable usar fuerza bruta y cu\'al es el comportamiento real de la programaci\'on din\'amica en escenarios m\'as exigentes.

Este tipo de comparaci\'on no busca ser un aporte novedoso a la literatura cient\'ifica, sino una herramienta de aprendizaje. A trav\'es de este informe, se espera poner en pr\'actica los conocimientos adquiridos en el curso, como el an\'alisis de eficiencia, la complejidad algor\'itmica y el dise\~no de experimentos.


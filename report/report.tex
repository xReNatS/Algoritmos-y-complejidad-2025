\documentclass[11pt,spanish]{article} % Tipo y tamaño de letra del documento.


\usepackage[utf8]{inputenc}
\usepackage{subfiles}
\usepackage{biblatex}
\addbibresource{references.bib}
\usepackage{multicol}
\usepackage{amsfonts}
\usepackage{blindtext}
\usepackage{mathrsfs}
\usepackage{amsmath}
\usepackage{siunitx}
\usepackage{centernot}
\usepackage[shortlabels]{enumitem}
\usepackage{subfig}
\usepackage{datetime}
\usepackage{listingsutf8}
\usepackage[spanish]{babel}
\usepackage{tikz}
\usepackage{hyperref}
\usepackage[vlined,ruled,linesnumbered]{algorithm2e}
\usepackage{listings}
\usepackage{float}
\usepackage{url}
\usepackage{csquotes}
\usepackage{fourier} %font
\usepackage[top=2cm, bottom=2cm, left=2.5cm, right=2.5cm]{geometry}
\usepackage{pgfplots}
\usepackage{fancyhdr}
\usepackage{mdframed}
\usepackage{tikzducks}
\usepackage[nameinlink]{cleveref}
\usepackage{epigraph} 

\pgfplotsset{compat=1.18}

\usetikzlibrary{shapes.arrows, shapes.geometric, arrows.meta,angles,quotes,positioning,arrows,fit,quotes,calc}
\tikzset{>=latex} 

\setlength\algomargin{1em} 
\SetFuncSty{sc} 
\SetCommentSty{em} 


\Crefname{figure}{Fig.}{Figs.}
\newcommand\crefrangeconjunction{--}
\Crefname{table}{Tabla}{Tablas}
\Crefname{subsubsection}{Subsubsec.}{Subsubsections}
\Crefname{subsection}{Subsec.}{Subsections}
\Crefname{section}{Sec.}{Sections}
\Crefname{equation}{eq.}{eqs.}
\crefname{thm}{Theorem}{theorems}
\Crefname{thm}{Theorem}{Theorems} 

\definecolor{algoco}{rgb}{0,0.4,1}

\hypersetup{
  colorlinks=true,
  linkcolor=algoco,
  citecolor=blue,
  urlcolor=blue,
}

\lstset{
extendedchars=true
inputencoding=utf8/latin1,
basicstyle=\footnotesize\sffamily\color{black},
commentstyle=\slshape \color{gray},
numbers=left,
numbersep=10pt,
numberstyle=\tiny\color{red!80!black},
keywordstyle=\color{red!80!magenta},
showspaces=false,
showstringspaces=false,
stringstyle=\color{cyan!80!black},
tabsize=2,
literate={á}{{\'a}}1 {é}{{\'e}}1 {í}{{\'i}}1 {ó}{{\'o}}1 {ú}{{\'u}}1,
frame = single, 
numbers = none,
float, floatplacement = ht, captionpos = b,
xleftmargin = 2em, xrightmargin = 2em, 
}

\newcommand{\ub}[1]{\underbrace{#1}}
\newcommand\tcm{\textcolor{magenta}}
\newcommand\tca{\textcolor{algoco}}

\setlength\epigraphwidth{.7\textwidth} 

\newcommand{\tnum}{2 y 3} % reemplace 2 por el número de la tarea
\newcommand{\sem}{2024-2} % reemplace 2024-2 por el semestre correspondiente
\newcommand{\campus}{Casa Central \\ Valparaíso} % reemplace Casa Central por el campus correspondiente
\newcommand{\rolusm}{202573100-1} % reemplace 2025073100-1 por su rol
\newcommand{\namestudent}{Al Goritmo Pérez} % reemplace Al Goritmo Pérez por su nombre

\headheight=14pt
\linespread{1.3}
\author{\namestudent}
\pagestyle{fancy}
\fancyhf{}%
\fancyfoot[R]{ \namestudent \\ \rolusm}
\fancyfoot[L]{Campus \campus} 
\fancyfoot[C]{\thepage}
\rhead{\sem}
\lhead{INF-221}
\renewcommand{\headrulewidth}{0.4pt}
\renewcommand{\footrulewidth}{0.4pt}
\newbool{programs}
\boolfalse{programs}
\chead{REPORTE TAREA \tnum~}



\title{
  \huge
  \textbf{REPORTE TAREA \tnum~ \\ ALGORITMOS Y COMPLEJIDAD} \\[1ex]
  \emph{\textquote{Más allá de la notación asintótica: Análisis experimental de algoritmos de ordenamiento y multiplicación de matrices.}} 
  }

  
\date{
  \small
  \today\\
  \currenttime
}




\begin{document}
\maketitle
\thispagestyle{fancy} 
\vspace{-1.0\baselineskip}




\begin{abstract}
  \textit{ 
    Este informe compara dos formas de resolver el problema de encontrar diferencias entre secuencias: una usando fuerza bruta y otra con programación dinámica. Para ello, se implementaron ambos algoritmos y se probaron con conjuntos de datos generados automáticamente, midiendo su tiempo de ejecución y uso de memoria.

Los resultados muestran que la programación dinámica es mucho más eficiente y escalable, especialmente cuando el número de casos crece, mientras que la fuerza bruta rápidamente se vuelve inviable. Esto demuestra la importancia de elegir el enfoque correcto al momento de diseñar soluciones algorítmicas. Además, el trabajo permitió validar en la práctica lo estudiado en teoría, y deja espacio para seguir explorando mejoras en este tipo de problemas.



  }
     
\end{abstract}

\setcounter{tocdepth}{1}
\tableofcontents


\newpage
\section{Introducción}
El an\'alisis y dise\~no de algoritmos es una de las bases fundamentales en la formaci\'on de estudiantes en Ciencias de la Computaci\'on. Comprender c\'omo se comportan distintos algoritmos en la pr\'actica permite tomar mejores decisiones sobre cu\'ando y por qu\'e utilizar uno u otro. En este trabajo se busca evaluar el rendimiento de dos enfoques para resolver un problema cl\'asico: encontrar las diferencias entre dos secuencias.

El problema de encontrar las diferencias entre dos secuencias consiste en identificar las partes no coincidentes entre ambas, resaltando al mismo tiempo las similitudes. Para lograr esto, se suele buscar la subsecuencia común más larga (LCS), y luego, con base en esa referencia, segmentar las porciones que no coinciden. Esta tarea es útil en áreas como el control de versiones, la comparación de textos o secuencias biológicas, donde es importante entender cómo y dónde dos cadenas difieren.

Para ello, se comparan dos estrategias distintas: una implementaci\'on por fuerza bruta, que explora todas las posibles subsecuencias para encontrar la subsecuencia com\'un m\'as larga, y otra basada en programaci\'on din\'amica, que utiliza una soluci\'on eficiente construyendo una tabla de subproblemas. Ambos enfoques fueron implementados y puestos a prueba en diferentes casos.

El objetivo principal de este informe es analizar cu\'ales son las diferencias pr\'acticas entre ambos algoritmos, en t\'erminos de tiempo de ejecuci\'on y uso de memoria. Para ello, se desarroll\'o una serie de experimentos que permiten observar hasta qu\'e punto es viable usar fuerza bruta y cu\'al es el comportamiento real de la programaci\'on din\'amica en escenarios m\'as exigentes.

Este tipo de comparaci\'on no busca ser un aporte novedoso a la literatura cient\'ifica, sino una herramienta de aprendizaje. A trav\'es de este informe, se espera poner en pr\'actica los conocimientos adquiridos en el curso, como el an\'alisis de eficiencia, la complejidad algor\'itmica y el dise\~no de experimentos.



\newpage
\section{Implementaciones}
\begin{mdframed}
    \textbf{La extensión máxima para esta sección es de 1 página.}
\end{mdframed}

Aquí deben explicar la estructura de sus programas haciendo referencias a los archivos y funciones de su entrega. No adjunte código en esta sección.
\begin{mdframed}
    Si se utiliza algún código, idea, o contenido extraído de otra fuente, este \textbf{debe} ser citado en el lugar exacto donde se utilice, en lugar de mencionarlo al final del informe.
\end{mdframed}

\newpage
\section{Experimentos}
\begin{mdframed}
    \textbf{La extensión máxima para esta sección es de 6 página.}
\end{mdframed}


\epigraph{``\textit{Non-reproducible single occurrences are of no significance to
science.}''}{---\citeauthor{popper2005logic},\citeyear{popper2005logic} \cite{popper2005logic}}
En la sección de Experimentos, es fundamental detallar la infraestructura utilizada para asegurar la reproducibilidad de los resultados, un principio clave en cualquier experimento científico. Esto implica especificar tanto el hardware (por ejemplo, procesador Intel Core i7-9700K, 3.6 GHz, 16 GB RAM DDR4, almacenamiento SSD NVMe) como el entorno software (sistema operativo Ubuntu 20.04 LTS, compilador g++ 9.3.0, y cualquier librería relevante). Además, se debe incluir una descripción clara de las condiciones de entrada, los parámetros utilizados y los resultados obtenidos, tales como tiempos de ejecución y consumo de memoria, que permitan a otros replicar los experimentos en entornos similares. \textit{La replicabilidad es un aspecto crítico para validar los resultados en la investigación científica computacional} \cite{inbookFonseca}.

\subsection{Dataset (casos de prueba)}
Los casos de prueba que fueron utilizados cubren las cuatro dimensiones especificadas para cada problema: tamaño de entrada (\(n\)), tipo de estructura o disposición (\(t\)), dominio de valores (\(d\)) y réplica de muestra (\(m\)). A continuación se comentan ambos grupos de casos.

\subsubsection{Multiplicación de matrices}
\begin{itemize}
  \item \textbf{Tamaños (\(n\))}: \(\{2^4,\,2^6,\,2^8,\,2^{10}\}\).  
    Estos valores permiten medir cómo escalan el tiempo (\(O(n^3)\) vs. \(O(n^{\log_2 7})\)) y el uso de memoria con el crecimiento de la dimensión.
  \item \textbf{Tipos (\(t\))}: \{\emph{densa}, \emph{diagonal}, \emph{dispersa}\}.  
    Cada estructura permite evaluar el rendimiento en escenarios de cómputo completo, optimización para ceros en diagonal y aprovechamiento de la dispersión de datos.
    \begin{itemize}
      \item \emph{Densa}: coeficientes aleatorios no nulos, representa el peor caso de cómputo completo.  
      \item \emph{Diagonal}: matrices con solo la diagonal principal distinta de cero, evalúa cómo afectan las optimizaciones en cero.  
      \item \emph{Dispersa}: pocos elementos no nulos distribuidos aleatoriamente, comprueba si el overhead de recursión amortiza el cálculo en estructuras crestas.  
    \end{itemize}
  \item \textbf{Dominio (\(d\))}: \(\{D0,\,D10\}\), con  
    \(D0=\{0,1\}\) y \(D10=\{0,\dots,9\}\).  
    Esta variación confirma que los algoritmos mantienen su comportamiento sin depender de la amplitud del dominio.
  \item \textbf{Muestras (\(m\))}: \(\{a,b,c\}\).  
    Tres réplicas para cada combinación permiten calcular varianza y construir intervalos de confianza de tiempo y memoria.
  \item \textbf{Generación}: los archivos fueron producidos con \texttt{matrix\_generator.py} 
  
  (en \texttt{code/matrix\_multiplication/scripts/}), de manera que se pueda reproducir lo experimentado.
\end{itemize}

\subsubsection{Ordenamiento}
\begin{itemize}
  \item \textbf{Tamaños (\(n\))}: \(\{10^1,\,10^3,\,10^5,\,10^7\}\).  
    Con ellos se mide el desempeño desde vectores muy pequeños hasta grandes, comparando \(O(n\log n)\) frente a \(O(n^2)\).
  \item \textbf{Tipos (\(t\))}: \{\emph{aleatorio}, \emph{ascendente}, \emph{descendente}\}.  
    Cada disposición corresponde a caso promedio, mejor y peor, permitiendo analizar la robustez de los algoritmos.
  \item \textbf{Dominio (\(d\))}: \(\{D1,\,D7\}\), donde  
    \(D1=\{0,1\}\) y \(D7=\{0,\dots,10^7\}\).  
    Esto asegura que las comparaciones no se vean afectadas por la dispersión de valores.
  \item \textbf{Muestras (\(m\))}: \(\{a,b,c\}\).  
    Tres instancias aleatorias para cada caso permiten estimar la variabilidad estadística de tiempos y consumos.
  \item \textbf{Generación}: los vectores fueron creados con \texttt{array\_generator.py} (en \texttt{code/sorting/scripts/}), asegurando que las pruebas coincidan con el repositorio oficial.
\end{itemize}


\subsection{Resultados}
A continuación se presentan y comentan las gráficas más representativas obtenidas tanto para los algoritmos de multiplicación de matrices como para los de ordenamiento. Cada bloque inicia con una descripción general de qué mide cada figura y termina con las principales observaciones.

\subsubsection{Multiplicación de matrices}

Para este experimento consideramos matrices de tamaños \(n = 2^4,2^6,2^8,2^{10}\) en los tres tipos de estructura (densa, diagonal y dispersa), midiendo tiempo de ejecución y memoria consumida por los algoritmos Naive y Strassen.

\begin{figure}[H]
    \centering
    \begin{minipage}[t]{1\textwidth}
        \includegraphics[width=\textwidth]{../code/matrix_multiplication/data/plots/memoria_vs_algoritmo.png}
     \end{minipage}%
    \caption{}
    \label{fig:scatterplot_3}
\end{figure}

\begin{figure}[H]
    \centering
    \begin{minipage}[t]{1\textwidth}
        \includegraphics[width=\textwidth]{../code/matrix_multiplication/data/plots/memoria_vs_algoritmo_densa.png}
     \end{minipage}%
    \caption{}
    \label{fig:scatterplot_3}
\end{figure}

\begin{figure}[H]
    \centering
    \begin{minipage}[t]{1\textwidth}
        \includegraphics[width=\textwidth]{../code/matrix_multiplication/data/plots/memoria_vs_algoritmo_diagonal.png}
     \end{minipage}%
    \caption{}
    \label{fig:scatterplot_3}
\end{figure}

\begin{figure}[H]
    \centering
    \begin{minipage}[t]{1\textwidth}
        \includegraphics[width=\textwidth]{../code/matrix_multiplication/data/plots/memoria_vs_algoritmo_dispersa.png}
     \end{minipage}%
    \caption{}
    \label{fig:scatterplot_3}
\end{figure}

\begin{figure}[H]
    \centering
    \begin{minipage}[t]{1\textwidth}
        \includegraphics[width=\textwidth]{../code/matrix_multiplication/data/plots/tiempo_vs_algoritmo.png}
     \end{minipage}%
    \caption{}
    \label{fig:scatterplot_3}
\end{figure}

\begin{figure}[H]
    \centering
    \begin{minipage}[t]{1\textwidth}
        \includegraphics[width=\textwidth]{../code/matrix_multiplication/data/plots/tiempo_vs_algoritmo_diagonal.png}
     \end{minipage}%
    \caption{}
    \label{fig:scatterplot_3}
\end{figure}

\begin{figure}[H]
    \centering
    \begin{minipage}[t]{1\textwidth}
        \includegraphics[width=\textwidth]{../code/matrix_multiplication/data/plots/tiempo_vs_algoritmo_dispersa.png}
     \end{minipage}%
    \caption{}
    \label{fig:scatterplot_3}
\end{figure}


\begin{itemize}
  \item \textbf{Memoria utilizada.}  
    \begin{itemize}
      \item El algoritmo Naive crece aproximadamente como \(O(n^2)\), reservando solo el espacio necesario para el producto final y unas pocas variables auxiliares.
      \item Strassen consume más memoria a medida que aumenta \(n\); esto se debe a las matrices temporales que crea en cada recursión, con un crecimiento cercano a \(O(n^{\log_2 7})\).  
    \end{itemize}

  \item \textbf{Tiempo de ejecución.}  
    \begin{itemize}
      \item El método Naive muestra un comportamiento cúbico \(O(n^3)\), con un aumento de tiempo muy acusado al duplicar la dimensión de la matriz.  
      \item Strassen reduce el tiempo efectivo, correspondiendo a su complejidad \(O(n^{2.807})\). La mejora es más evidente en tamaños grandes (\(n=2^8,2^{10}\)).  
      \item En matrices dispersas, ambos algoritmos tardan menos, pero la ventaja de Strassen permanece, aunque el overhead de recursión amortiza menos la ganancia para dimensiones pequeñas.  
    \end{itemize}
\end{itemize}

\begin{figure}[H]
    \centering
    \begin{minipage}[t]{1\textwidth}
        \includegraphics[width=\textwidth]{../code/sorting/data/plots/memoria_vs_algoritmo.png}
     \end{minipage}%
    \caption{}
    \label{fig:scatterplot_3}
\end{figure}

\begin{figure}[H]
    \centering
    \begin{minipage}[t]{1\textwidth}
        \includegraphics[width=\textwidth]{../code/sorting/data/plots/memoria_vs_algoritmo_aleatorio.png}
     \end{minipage}%
    \caption{}
    \label{fig:scatterplot_3}
\end{figure}

\begin{figure}[H]
    \centering
    \begin{minipage}[t]{1\textwidth}
        \includegraphics[width=\textwidth]{../code/sorting/data/plots/memoria_vs_algoritmo_ascendente.png}
     \end{minipage}%
    \caption{}
    \label{fig:scatterplot_3}
\end{figure}

\begin{figure}[H]
    \centering
    \begin{minipage}[t]{1\textwidth}
        \includegraphics[width=\textwidth]{../code/sorting/data/plots/memoria_vs_algoritmo_descendente.png}
     \end{minipage}%
    \caption{}
    \label{fig:scatterplot_3}
\end{figure}

\begin{figure}[H]
    \centering
    \begin{minipage}[t]{1\textwidth}
        \includegraphics[width=\textwidth]{../code/sorting/data/plots/tiempo_vs_algoritmo.png}
     \end{minipage}%
    \caption{}
    \label{fig:scatterplot_3}
\end{figure}

\begin{figure}[H]
    \centering
    \begin{minipage}[t]{1\textwidth}
        \includegraphics[width=\textwidth]{../code/sorting/data/plots/tiempo_vs_algoritmo_aleatorio.png}
     \end{minipage}%
    \caption{}
    \label{fig:scatterplot_3}
\end{figure}

\begin{figure}[H]
    \centering
    \begin{minipage}[t]{1\textwidth}
        \includegraphics[width=\textwidth]{../code/sorting/data/plots/tiempo_vs_algoritmo_ascendente.png}
     \end{minipage}%
    \caption{}
    \label{fig:scatterplot_3}
\end{figure}

\begin{figure}[H]
    \centering
    \begin{minipage}[t]{1\textwidth}
        \includegraphics[width=\textwidth]{../code/sorting/data/plots/tiempo_vs_algoritmo_descendente.png}
     \end{minipage}%
    \caption{}
    \label{fig:scatterplot_3}
\end{figure}

\subsubsection{Ordenamiento}

Probamos cuatro algoritmos de ordenamiento sobre vectores de \(n=10^5,10^6,10^7\) elementos, en tres tipos de entrada (aleatoria, ascendente y descendente). Medimos tiempo de ejecución y memoria ocupada.

\begin{itemize}
  \item \textbf{Memoria utilizada.}  
    \begin{itemize}
      \item \texttt{sortArray} (\texttt{std::sort}) aloca internamente un buffer proporcional a \(n\), por lo que su consumo escala linealmente y es el mayor de los cuatro.  
      \item \texttt{MergeSort} también requiere \(O(n)\) espacio adicional para sus fusiones, aunque con menor constante que \texttt{std::sort}.  
      \item \texttt{QuickSort} utiliza espacio de pila \(O(\log n)\), prácticamente constante en comparación.  
      \item \texttt{SelectionSort} sólo emplea variables escalares, manteniéndose casi en \(O(1)\).  
    \end{itemize}
 
  
  \item \textbf{Tiempo de ejecución.}  
    \begin{itemize}
      \item \texttt{SelectionSort} evidencia su \(O(n^2)\): con \(n=10^7\) alcanza tiempos del orden de \(10^6\) ms, muy superior al resto.  
      \item \texttt{MergeSort} y \texttt{QuickSort}, ambos \(O(n\log n)\), crecen de forma similar; \texttt{QuickSort} es ligeramente más rápido en entradas aleatorias.  
      \item \texttt{std::sort} combina introsort y heapsort, mostrando el mejor desempeño en todos los casos, especialmente en entradas ya ordenadas (donde QuickSort puro puede degradarse).  
      \item La variación entre aleatorio, ascendente y descendente es mínima para Merge y \texttt{std::sort}, mientras que QuickSort sufre un pequeño pico en orden descendente si no se usa pivote aleatorio.  
    \end{itemize}
\end{itemize}

Para le generación de gráficos, se utilizó los scripts "plotgenerator.py" de ambos tipos de algoritmos, que fueron creados mediante los resultados obtenidos en la carpeta de "measurements" con el archivo "mediciones.csv".





\newpage
\section{Conclusiones}
\begin{mdframed}
    \textbf{La extensión máxima para esta sección es de 1 página.}
\end{mdframed}

La conclusión de su informe debe enfocarse en el resultado más importante de su trabajo. No se trata de repetir los puntos ya mencionados en el cuerpo del informe, sino de interpretar sus hallazgos desde un nivel más abstracto. En lugar de describir nuevamente lo que hizo, muestre cómo sus resultados responden a la necesidad planteada en la introducción.

\begin{itemize}
    \item  No vuelva a describir lo que ya explicó en el desarrollo del informe. En cambio, interprete sus resultados a un nivel superior, mostrando su relevancia y significado.
    \item Aunque no debe repetir la introducción, la conclusión debe mostrar hasta qué punto logró abordar el problema o necesidad planteada en el inicio. Reflexione sobre el éxito de su análisis o experimento en relación con los objetivos propuestos.
    \item No es necesario restablecer todo lo que hizo (ya lo ha explicado en las secciones anteriores). En su lugar, centre la conclusión en lo que significan sus resultados y cómo contribuyen al entendimiento del problema o tema abordado.
    \item No deben centrarse en sí mismos o en lo que hicieron durante el trabajo (por ejemplo, evitando frases como "primero hicimos esto, luego esto otro...").
    \item Lo más importante es que no se incluyan conclusiones que no se deriven directamente de los resultados obtenidos. Cada afirmación en la conclusión debe estar respaldada por el análisis o los datos presentados. Se debe evitar extraer conclusiones generales o excesivamente amplias que no puedan justificarse con los experimentos realizados.
\end{itemize}


\newpage

\newpage
\appendix


\section{Apéndice 1}
Aquí puede agregar tablas, figuras u otro material que no se incluyó en el cuerpo principal del documento, ya que no constituyen elementos centrales de la tarea. Si desea agregar material adicional que apoye o complemente el análisis realizado, puede hacerlo en esta sección.

\begin{mdframed} 
    Esta sección es solo para material adicional. El contenido aquí no será evaluado directamente, pero puede ser útil si incluye material que será referenciado en el cuerpo del documento. Por lo tanto, asegúrese de que cualquier elemento incluido esté correctamente referenciado y justificado en el informe principal.
 \end{mdframed}


 
\printbibliography

\end{document}



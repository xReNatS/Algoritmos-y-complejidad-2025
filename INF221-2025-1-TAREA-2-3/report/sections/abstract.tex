Este informe compara dos formas de resolver el problema de encontrar diferencias entre secuencias: una usando fuerza bruta y otra con programación dinámica. Para ello, se implementaron ambos algoritmos y se probaron con conjuntos de datos generados automáticamente, midiendo su tiempo de ejecución y uso de memoria.

Los resultados muestran que la programación dinámica es mucho más eficiente y escalable, especialmente cuando el número de casos crece, mientras que la fuerza bruta rápidamente se vuelve inviable. Esto demuestra la importancia de elegir el enfoque correcto al momento de diseñar soluciones algorítmicas. Además, el trabajo permitió validar en la práctica lo estudiado en teoría, y deja espacio para seguir explorando mejoras en este tipo de problemas.


